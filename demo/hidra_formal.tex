\documentclass[a4paper,11pt]{article}
\usepackage[T1]{fontenc}
\usepackage[utf8]{inputenc}
\usepackage[spanish]{babel}
\usepackage{latexsym}               % agrega algunos símbolos
\usepackage{amssymb}                % símbolos especiales
\usepackage{amsmath, amsthm}
\usepackage{hyperref}
\usepackage{ccfonts}
\usepackage{titlesec}
\usepackage{titling}
\usepackage{verbatim}
\usepackage[margin=1.5in]{geometry}

\hypersetup{
    colorlinks=true,
    linkcolor=magenta,
}
\setlength{\parindent}{0pt}
\titleformat*{\section}{\large\centering}

\newcommand{\seb}{\subseteq}
\newcommand{\sqsm}{\sqsubset}
\newcommand{\bac}{\backslash}
\newcommand{\matN}{\mathbb{N}}
\newcommand{\matR}{\mathbb{R}}
\newcommand{\sep}{\;|\;}
\newcommand{\nR}{\!\!\not\mathrel{R}\!\!}
\newcommand{\nH}{\!\not\mathrel{H}\!}
\newcommand{\Mod}[1]{\ (\mathrm{mod}\ #1)}
\newcommand{\mA}{\mathcal{A}}
\newcommand{\mH}{\mathcal{H}}
\newcommand{\lag}{\langle}
\newcommand{\rag}{\rangle}

\setlength{\droptitle}{-10em}

\theoremstyle{definition}
\newtheorem{corolary}{Corolario}
\newtheorem{lemma}{Lema}
\newtheorem{proposition}{Proposición}
\newtheorem{theorem}{Teorema}
\newtheorem{example}{Ejemplo}
\newtheorem{definition}{Definición}


\hypersetup{
    colorlinks=true,
    linkcolor=blue,
    filecolor=magenta,
    urlcolor=cyan,
}

\title{Demostración formal de la terminación del Juego de la Hidra.}
\author{Lucas Polymeris}

\begin{document}
\maketitle

Asumimos familiaridad con las reglas del {\em Juego de la Hidra},
introducido por L. Kirby y J. Paris en \cite{hydra}. En el siguiente
\href{https://slate.com/human-interest/2014/06/hydra-game-an-example-of-a-counterintuitive-mathematical-result.html}{enlace}
se puede encontrar una buena explicación de estas. También
se puede jugar al juego con el \href{https://github.com/Average-user/hydra-game#readme}{programa} del autor. Adoptamos la notación
(como es usual en combinatoria) de representar el conjunto $\{1,2,\dots,n\}$
por $[n]$.\\

La estrategia para la demostración es como sigue. Primero formalizamos la noción
de árbol ``plantado'', definimos un orden sobre estos, y definimos las
hidras como representantes de la clase de isomorfismos de estos árboles. Mostramos que
este orden restringido a las hidras es {\em bien fundado}, es decir, que toda sucesión estrictamente
decreciente de hidras debe ser finita, así como en $\matN$, y como no
es el caso en $\mathbb{Q}_{\geq 0}$. Por último, mostramos que si $\alpha'$ es una hidra producida
por una jugada válida desde $\alpha$, entonces $\alpha' < \alpha$. Concluyendo
entonces, que si se sigue jugando, siempre llegará un momento en el que
se llegue a la hidra de solo un nodo, donde no hay jugada posible.

\begin{definition}
  Definimos un árbol ``plantado'' finito inductivamente como sigue:
  \begin{itemize}
  \itemsep0em
  \item $\lag\rag$ es el árbol de solo un nodo.
  \item Si $t_1,\dots,t_n$ son árboles plantados, entonces $\lag t_1,\dots,t_n\rag$ es un árbol-plantado.
    Decimos que $t_1,\dots,t_n$ son los sub-árboles directos de $\lag t_1,\dots,t_n \rag$.
  \end{itemize}
  Consideramos $\mathcal{A}$ el conjunto de todos los árboles finitos, y si
  $\alpha \in \mA$ con $\alpha = \lag t_1,\dots,t_n \rag$ entendemos
   $\alpha[i] = t_i$.
\end{definition}

\begin{definition}
  Definimos el orden $(\mA,\leq)$ recursivamente. Sean $\alpha,\beta \in \mA$,
  con $\alpha = \lag a_1,\dots,a_p \rag$ y $\beta = \lag b_1,\dots,b_q \rag$. Considere
  $$I := \{i \in [\min\{p,q\}] : a_i \neq b_i\}$$ Si $p \leq q$ e $I$ es vacío,
  entonces definimos $\alpha \leq \beta$.
  Si $I$ es no vacío, sea $i_0 := \min(I)$, luego $\alpha \leq \beta$ si
  $a_{i_0} < b_{i_0}$.
\end{definition}

Es claro que en $\mA$ hay muchas copias de un ``mismo'' árbol en el sentido de isomorfismo. Por ejemplo los
árboles $\lag\lag\rag,\lag\lag\rag,\lag\rag\rag\rag$ y $\lag\lag\lag\rag,\lag\rag\rag,\lag\rag\rag$ son isomorfos,
sin embargo, bajo el orden definido tenemos
$\lag\lag\rag,\lag\lag\rag,\lag\rag\rag\rag < \lag\lag\lag\rag,\lag\rag\rag,\lag\rag\rag$, lo cual causaría
problemas más adelante. Además, es claro
que en lo que al juego de la hidra respecta, dos árboles isomorfos son la ``misma'' hidra.
En esencia, nos gustaría que las hidras
sean los representantes de la clase de isomorfismo, que tienen sus hijos ordenados
de mayor a menor según el orden definido, los hijos de hijos también y así. Lo
cual nos lleva a la siguiente definición.

\begin{definition}
  Definimos las hidras recursivamente.
  \begin{itemize}
    \itemsep0em
    \item $\lag\rag$ es la hidra de solo un nodo.
    \item Si $r_1,\dots,r_n$ son hidras tal que $r_1 \geq \dots \geq r_n$, entonces
      $\lag r_1,\dots,r_n \rag$ es una hidra.
  \end{itemize}
  Llamamos por $\mH$ el conjunto de todas las hidras, donde es claro que $\mH \seb \mA$.\\
  Llamamos a los sub-árboles directos de una hidra, sus hijos. Además
  definimos la altura $h(\alpha)$ de una hidra $\alpha$, como la
  cantidad de paréntesis izquierdos al comienzo de $\alpha$, antes de encontrarse
  con uno derecho.
\end{definition}

Debiera ser claro de esta definición, que para todo árbol existe
una hidra isomorfa a este, y que por lo tanto, no nos ``faltan'' hidras por
considerar.
\begin{comment}
\begin{lemma}
  Si $\alpha$ y $\beta$ son hidras tal que $h(\alpha) < h(\beta)$, entonces
  $\alpha < \beta$.
\end{lemma}
\begin{proof}
  Procedemos por inducción sobre $h(\alpha)$.\\
  $h(\alpha) = 1 \Rightarrow \alpha = \lag\rag$ y el lema se cumple. Asuma
  $h(\alpha) > 1$ y que el lema se cumple para $h(\alpha)-1$. Luego
  $\alpha = \lag a_1,\dots,a_p\rag$ y $\beta = \lag b_1,\dots,b_q\rag$, donde
  por definición de altura es claro que $h(a_1) = h(\alpha)-1 < h(\beta)-1 = h(b_1)$ y por
  lo tanto, por hipótesis de inducción $a_1 < b_1 \Rightarrow \alpha < \beta$.
\end{proof}
\end{comment}
\begin{lemma}
  Sea $(\alpha_i)_{i \in \matN}$ una sucesión de hidras tal que
  $$
  \alpha_1 \geq \alpha_2 \geq \dots
  $$
  luego debe ser que $\alpha$ se estabiliza. Es decir, existe $i_0$ tal
  que $\alpha_{i_0} = \alpha_{i_0 + k}$ para todo $k \in \matN$.
\end{lemma}
\begin{proof}
  Procedemos por inducción sobre $H := \min\{h(\alpha_i) : i \in \matN\}$. Si
  $H = 1$, debe ser que $\alpha_1= \lag\rag$, y en consecuencia la sucesión se estabiliza
  en $\alpha_1$.\\
  Suponga que $H>1$ y suponga que el lema se cumple para valores menores a $H$.
  Como $H>1$, $\lag\rag$ no aparece en la sucesión y por lo tanto
  $$
  f_{1}(i) := \alpha_i[1]
  $$
  es bien definida ya que todo $\alpha_i$ tiene al menos un hijo. Además
  es claro que $h(f_1(i)) = h(\alpha_i) - 1$. Por lo tanto, por hipótesis de inducción
  $f_1$ se estabiliza, es decir, existe $t_1$ tal que para todos excepto finitos
  de los naturales $i$ se tiene $f_1(i) = t_1$. Esto implica que para estos $i$
  $$
  \alpha_i = \lag t_1,\dots,t_1,r_i,r_{i+1}\dots\rag
  $$
  Es decir, que $\alpha_i$ comienza con una ``tira'' de $t_1$'s. De todos estos
  $\alpha_i$, sea $K$ el largo de la tira de $t_1$'s más corta. Si fuera
  que para algún $i$, $\alpha_i$ consiste exactamente de esta tira de $t_1$'s de
  largo $K$, entonces es claro que $\alpha$ se estabiliza en tal $\alpha_i$ y el
  lema vale. Llamamos a este caso $(\ast)$.\\
  Si no, debe ser que a partir de cierto $i_1$, todos los $\alpha_i$, tienen
  un termino además de la tira de $t_1$'s de largo $K$, y por lo tanto la función
  $$
  f_2(i) = \alpha_{i_1 + i}[K+1]
  $$
  es bien definida. Es claro además, que para todo $i$, $h(f_2(i)) \leq h(t_1) < H$,
  y que por lo tanto, $f_2$ se estabiliza en algún valor $t_2$. Además,por construcción
  debe ser que $t_2 < t_1$.
  Iterando este proceso, formamos una sucesión de hidras
  $$
  t_1 > t_2 > \dots
  $$
  que por hipótesis de inducción debe estabilizarse. Pero esto no es posible, ya que
  cada término es estrictamente menor que el anterior, concluyendo que la sucesión
  debe ser finita. Como la única posibilidad de que no se pueda construir $t_{i+1}$
  es que en $t_i$ se de el caso $(\ast)$, concluimos que este debe darse para algún $i$,
  completando la demostración del lema.
\end{proof}

\begin{corolary}
  $(\mH,\leq)$ es un orden bien fundado.
\end{corolary}

Todavía no queda claro, en esta forma de representar una
hidra, cuales son las ``cabezas'' de esta. Para definir esto, nos
apoyamos en el hecho de que en un árbol plantado, existe un único
camino de la raíz a cada hoja. Lo cual motiva la siguiente definición.

\begin{definition}
  Dado $\alpha \in \mH$, decimos que $(x_0,x_1,\dots,x_n)$ es la
  dirección de una cabeza de $\alpha$ si es tal que
  $x_0 = \alpha$, $x_n = \lag\rag$, y para todo
  $i \in [n]$ se tiene que $x_{i}$ es hijo $x_{i-1}$. Decimos entonces
  que la cabeza representada tiene profundidad $n$.
\end{definition}

Nos gustaría definir el conjunto $R(\alpha)$ de las hidras
que pueden resultar de una jugada válida desde $\alpha$, sin embargo,
estas dependen de en que ``fase'' esté el juego, es decir, de cuantos
turnos hayan pasado. Sin embargo, esto no es problema, ya que podemos
simplemente considerar $R(\alpha)$ como la unión de las hidras
que resultan de $\alpha$ en un turno $i$ para todo $i \in \matN$. Esto
nos lleva a la siguiente definición.

\begin{definition}
  Dado $\alpha \in \mH$, denotamos por $R(\alpha)$ el conjunto
  de hidras que se pueden alcanzar desde $\alpha$ mediante a una
  jugada del Juego de la Hidra, para alguna fase $i \in \matN$. Note
  que sigue de las reglas del juego que $R(\lag\rag) = \{\lag\rag\}$.
\end{definition}

\begin{lemma}
  Si $\alpha \in \mH \setminus \{\lag\rag\}$, $\alpha' \in R(\alpha) \Rightarrow \alpha' < \alpha$.
\end{lemma}
\begin{proof}
  Procedemos por inducción por sobre la profundidad $p$ de la cabeza cortada. Si
  $p=1$, entonces la cabeza tiene dirección $(\alpha,\lag\rag)$ y en consecuencia
  $\alpha = \lag a_1,\dots,a_n,\lag\rag\rag$, para algunos $a_1,\dots,a_n \in \mH$,
  luego, por las reglas del juego $\alpha' = \lag a_1,\dots,a_n \rag < \alpha$.\\ Si
  $p=2$, la dirección es de la forma $(\alpha,x,\lag\rag)$, y por lo tanto $x$
  debe ser de la forma $\lag s_1,\dots,s_r,\lag\rag\rag$ y $\alpha$  de la forma
  $$
  \lag a_1,\dots,a_{i-1},\underbrace{x,\dots,x}_{k \text{ veces}},a_{i},\dots,a_n\rag
  $$
  y en consecuencia $\alpha'$ es de la forma
  $$
  \lag a_1,\dots,a_{i-1},\underbrace{x,\dots,x}_{k-1 \text{ veces}},y,\dots\rag
  $$
  donde $y$ es o bien una de las muchas
  copias de $\lag s_1,\dots,s_r\rag$, o $a_{j}$ con $i \leq j \leq n$. En cualquier
  caso $y < x$ y concluimos $\alpha' < \alpha$.\\
  Luego, suponga $p > 2$ y que el lema se cumple para $p-1$. Luego, debe
  ser que la dirección es de la forma $(\alpha,x_1,x_2,\dots,x_p)$ y entonces
  $\alpha$ es de la forma
  $$
  \lag a_1,\dots,a_{i-1},\underbrace{x_1,\dots,x_1}_{k \text{ veces}},a_{i},\dots,a_n\rag
  $$
   y como $p\geq 3$ se sigue por reglas del juego, que $\alpha'$ debe ser de la forma
  $$
  \lag a_1,\dots,a_{i-1},\underbrace{x_1,\dots,x_1}_{k-1 \text{ veces}},y,\dots\rag
  $$
  donde o bien $y = x_1'$, el resultado de cortar la cabeza con dirección $(x_1,\dots,x_p)$ de $x_1$,
  que como esta tiene profundidad $p-1$, se sigue por hipótesis de inducción
  que $x_1' < x_1$, y por lo tanto $\alpha' < \alpha$. O $ y = a_{j}$ con
  $i \leq j \leq n$, también concluyendo $\alpha' < \alpha$.
\end{proof}

\begin{theorem}
  Cualquier estrategia del Juego de la Hidra es ganadora.
\end{theorem}
\begin{proof}
  Sigue de los lemas anteriores.
\end{proof}


\bibliographystyle{abbrv}
\bibliography{refs}

\end{document}
